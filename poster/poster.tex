%%%%%%%%%%%%%%%%%%%%%%%%%%%%%%%%%%%%%%
% LaTeX poster template
% Created by Nathaniel Johnston
% August 2009
% http://www.nathanieljohnston.com/2009/08/latex-poster-template/
%%%%%%%%%%%%%%%%%%%%%%%%%%%%%%%%%%%%%%

\documentclass[final]{beamer}
\usepackage[scale=1.24]{beamerposter}
\usepackage{graphicx}			% allows us to import images

%-----------------------------------------------------------
% Define the column width and poster size
% To set effective sepwid, onecolwid and twocolwid values, first choose how many columns you want and how much separation you want between columns
% The separation I chose is 0.024 and I want 4 columns
% Then set onecolwid to be (1-(4+1)*0.024)/4 = 0.22
% Set twocolwid to be 2*onecolwid + sepwid = 0.464
%-----------------------------------------------------------

\newlength{\sepwid}
\newlength{\onecolwid}
\newlength{\twocolwid}
\newlength{\threecolwid}
\setlength{\paperwidth}{48in}
\setlength{\paperheight}{36in}
\setlength{\sepwid}{0.024\paperwidth}
\setlength{\onecolwid}{0.22\paperwidth}
\setlength{\twocolwid}{0.464\paperwidth}
\setlength{\threecolwid}{0.708\paperwidth}
\setlength{\topmargin}{-0.5in}
\usetheme{confposter}
\usepackage{exscale}

%-----------------------------------------------------------
% The next part fixes a problem with figure numbering. Thanks Nishan!
% When including a figure in your poster, be sure that the commands are typed in the following order:
% \begin{figure}
% \includegraphics[...]{...}
% \caption{...}
% \end{figure}
% That is, put the \caption after the \includegraphics
%-----------------------------------------------------------

\usecaptiontemplate{
\small
\structure{\insertcaptionname~\insertcaptionnumber:}
\insertcaption}

%-----------------------------------------------------------
% Define colours (see beamerthemeconfposter.sty to change these colour definitions)
%-----------------------------------------------------------

\setbeamercolor{block title}{fg=ngreen,bg=white}
\setbeamercolor{block body}{fg=black,bg=white}
\setbeamercolor{block alerted title}{fg=white,bg=dblue!70}
\setbeamercolor{block alerted body}{fg=black,bg=dblue!10}

%-----------------------------------------------------------
% Name and authors of poster/paper/research
%-----------------------------------------------------------

\title{Punitive Measures: Because Finding a Good Pun is its own Reword}
\author{GROUP C - Amir Kashipazha, Brandon Boylan-Peck, Cathlyn Stone, Kenneth Hunter Wapman, Shantanu Karnwal}
\institute{CSCI 5622 (Machine Learning), University of Colorado Boulder}

%-----------------------------------------------------------
% Start the poster itself
%-----------------------------------------------------------

\begin{document}
\begin{frame}[t]
	\begin{columns}[t] % the [t] option aligns the column's content at the top
		\begin{column}{\sepwid}\end{column}			% empty spacer column
		\begin{column}{\onecolwid}
		% \begin{alertblock}{Objectives}
		% 	{\Large
		% 		We worked on two tasks: pun detection (identifying whether a sentence is a pun or not) and pun location (identifying which word in a pun sentence 'makes' the pun)
		% 	}
		% \end{alertblock}

		\begin{block}{Introduction and Motivation}
			{\large
				A \textbf{Pun} is form of wordplay in which a word\textquotesingle s multiple meanings or similar sounding words are used in a comedical manner. A pun can appear in any context where language is used, and the intended effect of a pun often varies depending on the context.\\
					\vskip2ex
				There are many types of puns, but in our project we focused on these two types:\\
					\\
				
				\begin{itemize}
					\item {\textbf{Homographic}: A pun involving two words whose spellings are the same but have different meanings.\\
							\textit{I used to be a banker but I lost \textbf{interest}.}}
					\item {\textbf{Heterographic}: A pun involves two or more words which sound alike but are spelled differently.\\
							\textit{Two construction workers had a \textbf{stairing} contest.}}
				\end{itemize}
				\\

				If a sentence is not understood to be a pun, or if the pun is not understood or is misunderstood, meaning has been lost, making the accurate identification and interpretation of puns essential in any domain they appear in.\\
				\begin{itemize}
					% \item 
					\item If a machine could understand puns, or even generate puns, it could be used for the benefit of companies, to create more natural conversation, and even to assist in translation.
					\item The benefits of pun learning are outstanding, and even further, it would be a step towards better understanding and surpassing the ambiguity of much natural language.
				\end{itemize}
			}
		\end{block}
		\vskip2ex


		\begin{block}{Tasks}
			{\large
				We worked on two tasks: 
				\begin{itemize}
					% \item 
					\item {Pun \textbf{Detection}: determining whether a sentence is a pun or not}
					\item {Pun \textbf{Location}: identifying the word in a pun sentence which 'makes' the pun}
				\end{itemize}
				We used machine learning techniques to detect and locate a pun in a sentence. Along with that, we are also able to find out that if there is a pun, whether it is homographic or heterographic.\\
			}

		\end{block}

		\end{column}

\begin{column}{\sepwid}\end{column} % Empty spacer column

\begin{column}{\twocolwid} % Begin a column which is two columns wide (column 2)

\begin{columns}[t,totalwidth=\twocolwid] % Split up the two columns wide column

\begin{column}{\onecolwid}\vspace{-.6in} % The first column within column 2 (column 2.1)

    \begin{block}{Task 1 - Pun Detection}
        {\large Pun detection is a binary classification problem: given a sentence, determine whether or not that sentence contains a pun. Given the binary nature of the solution space and the highly non-linear input space, machine learning seems to be a natural choice to detect puns.\\
        We have used three approaches for pun detection - 
        \begin{itemize}
          \item {\textbf{Naive Bayes classifier (Baseline)}}
          \item {\textbf{Recurrent Neural Network}}
           \item {\textbf{Feature Engineering based classifier}: We used the Stochastic Gradient Descent algorithm while combining different features like lesk algorithm, homophones, homonyms, antonyms, idioms, part of speech tagging.}
        \end{itemize}
        }
      \end{block}



%----------------------------------------------------------------------------------------

\end{column} % End of column 2.1

\begin{column}{\onecolwid}\vspace{-.6in} % The second column within column 2 (column 2.2)

\begin{SCfigure}
\includegraphics[width=0.85\textwidth]{HomographicDetection.png}\\
\caption{Figure 1.1 - Pun Detection algorithm running on the Homographic Dataset.}
\end{SCfigure}
\\
\vspace{20mm}
\begin{SCfigure}
\includegraphics[width=0.85\textwidth]{HeterographicDetection.png}\\
\caption{Figure 1.2 - Pun Detection algorithm running on the Heterographic Dataset.}
\end{SCfigure}


%----------------------------------------------------------------------------------------

\end{column} % End of column 2.2

\end{columns} % End of the split of column 2 - any content after this will now take up 2 columns width

\begin{block}
    {.}
\end{block}

\begin{columns}[t,totalwidth=\twocolwid] % Split up the two columns wide column again

\begin{column}{\onecolwid} % The first column within column 2 (column 2.1)

%----------------------------------------------------------------------------------------
%	MATHEMATICAL SECTION
%----------------------------------------------------------------------------------------

\begin{block}{Task 2 - Pun Location}
        {\large The Pun locaton problem can be remodelled as: for each word in the sentence, determine the likelihood that that word is a pun given the rest of the sentence. Then, output the word which is most likely to be a pun based on the output of the previous search. \\
        We have used three approaches for pun detection - 
        \begin{itemize}
          \item {\textbf{Naive Bayes classifier (Baseline)}}
           \item {\textbf{Recurrent Neural Network} %: A homographic pun involves two words which are spelled the same but have different meanings.}
           }
          \item {\textbf{Sliding Window Based Classifier} : The sliding window classifier starts from the first word of the sentence and two words on either side to make a window of size five. Then it runs a max entropy classifier on this window, before sliding the window one word forward through the sentence. The block with highest entropy gives out the pun word.}
          \end{itemize}
          \\
        }
      \end{block}
      
%----------------------------------------------------------------------------------------

\end{column} % End of column 2.1

\begin{column}{\onecolwid} % The second column within column 2 (column 2.2)

%----------------------------------------------------------------------------------------
%	RESULTS
%----------------------------------------------------------------------------------------

\begin{SCfigure}
\includegraphics[width=0.85\textwidth]{HomographicLocation.png}\\
\caption{Figure 2.1 - Pun Location algorithm running on the Homographic Dataset.}
\end{SCfigure}
\\
\vspace{20mm}
\begin{SCfigure}
\includegraphics[width=0.85\textwidth]{HeterographicLocation.png}\\
\caption{Figure 2.2 - Pun Location algorithm running on the Heterographic Dataset.}
\end{SCfigure}

%----------------------------------------------------------------------------------------

\end{column} % End of column 2.2

\end{columns} % End of the split of column 2

\end{column} % End of the second column

\begin{column}{\sepwid}\end{column} % Empty spacer column


    \begin{column}{\sepwid}\end{column}			% empty spacer column
    \begin{column}{\onecolwid}
            
\begin{block}{UI for Pun Detection \& Location}
\begin{SCfigure}
\centering
\includegraphics[width=0.75\textwidth]{UI_1.jpg}\\
\caption{Figure 3.1 - Welcome Page.}
\end{SCfigure}
\\
\vspace{10mm}
\begin{SCfigure}
\centering
\includegraphics[width=0.75\textwidth]{UI_2.png}\\
\centering
\caption{Figure 3.2 - Analysis of the input sentence}
\end{SCfigure}
\end{block}
\vspace{20mm}
            \begin{block}{Challenges we faced}
                    
            \end{block}
            \vspace{20mm}
            \begin{block}{How well did we do?}
                    
            \end{block}
            
		    
    \end{column}

      		
 \end{columns}
\end{frame}
\end{document}

