\documentclass{article}

\usepackage[final]{nips_2017}

\usepackage[utf8]{inputenc} % allow utf-8 input
\usepackage[T1]{fontenc}    % use 8-bit T1 fonts
\usepackage{hyperref}       % hyperlinks
\usepackage{url}            % simple URL typesetting
\usepackage{booktabs}       % professional-quality tables
\usepackage{amsfonts}       % blackboard math symbols
\usepackage{nicefrac}       % compact symbols for 1/2, etc.
\usepackage{microtype}      % microtypography

\title{Pun Classification and Location}

% The \author macro works with any number of authors. There are two
% commands used to separate the names and addresses of multiple
% authors: \And and \AND.
%
% Using \And between authors leaves it to LaTeX to determine where to
% break the lines. Using \AND forces a line break at that point. So,
% if LaTeX puts 3 of 4 authors names on the first line, and the last
% on the second line, try using \AND instead of \And before the third
% author name.

\author{
	Shantanu Karnwal
	\texttt{shantanu.karnwal@colorado.edu} \\
	\And
	Amir Kashipazha
	\texttt{amirhossein.kashipazha@colorado.edu} \\
	\And
	Brandon Boylan-Peck
	\texttt{brbo9266@colorado.edu} \\
	\And
	Cathlyn Stone\\
	\texttt{cathlyn.stone@colorado.edu} \\
	\And
	Kenneth Hunter Wapman\\
	\texttt{kennethwapman@colorado.edu} \\
}

\begin{document}

\maketitle

\begin{abstract}
	Pun identification and location are challenging natural language processing 
	tasks. We implemented several algorithms for both, with results which 
	were comparable to those outlined in the SemEval conference which originally
	defined the tasks.
\end{abstract}

%%%%%%%%%%%%%%%%%%%%%%%%%%%%%%%%%%%%%%%%%%%%%%%%%%%%%%%%%%%%%%%%%%%%%%%%%%%%%%%
% Problem Description
%%%%%%%%%%%%%%%%%%%%%%%%%%%%%%%%%%%%%%%%%%%%%%%%%%%%%%%%%%%%%%%%%%%%%%%%%%%%%%%

\section{Overview}

\subsection{Pun Structure}

here we will talk about what puns are
\begin{itemize}

\item homographic
\item heterographic
\item other types

\end{itemize}

here we will talk about the problems we worked on.

\begin{center}
	homographic pun example
  % \url{https://cmt.research.microsoft.com/NIPS2017/}
\end{center}

\begin{center}
	heterographic pun example
  % \url{https://cmt.research.microsoft.com/NIPS2017/}
\end{center}

Please read carefully the instructions below and follow them
faithfully.

\subsection{Motivation}


\subsection{Tasks}

section references \ref{pun_detection}, \ref{pun_location}, and
\ref{conclusion} below.

%%%%%%%%%%%%%%%%%%%%%%%%%%%%%%%%%%%%%%%%%%%%%%%%%%%%%%%%%%%%%%%%%%%%%%%%%%%%%%%
% Pun Detection
%%%%%%%%%%%%%%%%%%%%%%%%%%%%%%%%%%%%%%%%%%%%%%%%%%%%%%%%%%%%%%%%%%%%%%%%%%%%%%%

\section{Pun Detection}
\label{pun_detection}

here we will talk about pun detection and our algorithms.

\subsection{Baseline}

\subsection{Algorithms}

\subsection{Feature Comparison}

%%%%%%%%%%%%%%%%%%%%%%%%%%%%%%%%%%%%%%%%%%%%%%%%%%%%%%%%%%%%%%%%%%%%%%%%%%%%%%%
% Pun Location
%%%%%%%%%%%%%%%%%%%%%%%%%%%%%%%%%%%%%%%%%%%%%%%%%%%%%%%%%%%%%%%%%%%%%%%%%%%%%%%
\section{Pun Location}
\label{pun_location}

\subsection{Baseline}

\subsection{Algorithms}

\subsection{Feature Comparison}

%%%%%%%%%%%%%%%%%%%%%%%%%%%%%%%%%%%%%%%%%%%%%%%%%%%%%%%%%%%%%%%%%%%%%%%%%%%%%%%
% Results
%%%%%%%%%%%%%%%%%%%%%%%%%%%%%%%%%%%%%%%%%%%%%%%%%%%%%%%%%%%%%%%%%%%%%%%%%%%%%%%
\section{Results}
\label{results}

\subsection{Pun Detection}
Here's how we think we did.
\subsubsection{Evaluation}
\subsubsection{Results}
here are our results. here's the baseline. here's semeval's

\subsubsection{Error Analysis}

\subsection{Pun Location}
Here's how we think we did.
\subsubsection{Evaluation}
f score etc etc
\subsubsection{Results}
here are our results. here's the baseline. here's semeval's
\subsubsection{Error Analysis}

%%%%%%%%%%%%%%%%%%%%%%%%%%%%%%%%%%%%%%%%%%%%%%%%%%%%%%%%%%%%%%%%%%%%%%%%%%%%%%%
% Conclusion
%%%%%%%%%%%%%%%%%%%%%%%%%%%%%%%%%%%%%%%%%%%%%%%%%%%%%%%%%%%%%%%%%%%%%%%%%%%%%%%

\section{Conclusion}
\label{conclusion}

\subsection{who did what}
\subsection{what went well}
\subsection{what we could have done better}

\subsubsection*{Acknowledgments}

here's where we acknowledge stuff

%%%%%%%%%%%%%%%%%%%%%%%%%%%%%%%%%%%%%%%%%%%%%%%%%%%%%%%%%%%%%%%%%%%%%%%%%%%%%%%
% References
%%%%%%%%%%%%%%%%%%%%%%%%%%%%%%%%%%%%%%%%%%%%%%%%%%%%%%%%%%%%%%%%%%%%%%%%%%%%%%%
\section*{References}

\medskip

\small

[1] Alexander, J.A.\ \& Mozer, M.C.\ (1995) Template-based algorithms
for connectionist rule extraction. In G.\ Tesauro, D.S.\ Touretzky and
T.K.\ Leen (eds.), {\it Advances in Neural Information Processing
  Systems 7}, pp.\ 609--616. Cambridge, MA: MIT Press.

[2] Bower, J.M.\ \& Beeman, D.\ (1995) {\it The Book of GENESIS:
  Exploring Realistic Neural Models with the GEneral NEural SImulation
  System.}  New York: TELOS/Springer--Verlag.

[3] Hasselmo, M.E., Schnell, E.\ \& Barkai, E.\ (1995) Dynamics of
learning and recall at excitatory recurrent synapses and cholinergic
modulation in rat hippocampal region CA3. {\it Journal of
  Neuroscience} {\bf 15}(7):5249-5262.

%%%%%%%%%%%%%%%%%%%%%%%%%%%%%%%%%%%%%%%%%%%%%%%%%%%%%%%%%%%%%%%%%%%%%%%%%%%%%%%
% Misc Style examples
%%%%%%%%%%%%%%%%%%%%%%%%%%%%%%%%%%%%%%%%%%%%%%%%%%%%%%%%%%%%%%%%%%%%%%%%%%%%%%%
\section{Style Stuff}
\verb+this is code-ish looking+ 

here's a percent: $\sim$$15\%$
\textbf{boldboldbold}
\emph{italics}

here's a fraction: \nicefrac{1}{4}

\paragraph{}This is paragraphed
\begin{verbatim}
   \citet{hasselmo} investigated\dots
\end{verbatim}
produces
\begin{quote}
  Hasselmo, et al.\ (1995) investigated\dots
\end{quote}

ref number:\ [4]

footnotes.\footnote{they go after the period}

\begin{figure}[h]
  \centering
  \fbox{\rule[-.5cm]{0cm}{4cm} \rule[-.5cm]{4cm}{0cm}}
  \caption{Sample figure caption.}
\end{figure}

Table~\ref{sample-table}.

use \verb+booktabs+ package for tables
\begin{center}
  \url{https://www.ctan.org/pkg/booktabs}
\end{center}

\begin{table}[t]
  \caption{Sample table title}
  \label{sample-table}
  \centering
  \begin{tabular}{lll}
    \toprule
    \multicolumn{2}{c}{Part}                   \\
    \cmidrule{1-2}
    Name     & Description     & Size ($\mu$m) \\
    \midrule
    Dendrite & Input terminal  & $\sim$100     \\
    Axon     & Output terminal & $\sim$10      \\
    Soma     & Cell body       & up to $10^6$  \\
    \bottomrule
  \end{tabular}
\end{table}

\end{document}
